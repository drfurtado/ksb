\documentclass[11pt,]{article}
\usepackage[margin=1in]{geometry}
\newcommand*{\authorfont}{\fontfamily{phv}\selectfont}
\usepackage[]{mathpazo}
\usepackage{abstract}
\renewcommand{\abstractname}{}    % clear the title
\renewcommand{\absnamepos}{empty} % originally center
\newcommand{\blankline}{\quad\pagebreak[2]}

\providecommand{\tightlist}{%
  \setlength{\itemsep}{0pt}\setlength{\parskip}{0pt}} 
\usepackage{longtable,booktabs}

\usepackage{parskip}
\usepackage{titlesec}
\titlespacing\section{0pt}{12pt plus 4pt minus 2pt}{6pt plus 2pt minus 2pt}
\titlespacing\subsection{0pt}{12pt plus 4pt minus 2pt}{6pt plus 2pt minus 2pt}

\usepackage{titling}
\setlength{\droptitle}{-.25cm}

%\setlength{\parindent}{0pt}
%\setlength{\parskip}{6pt plus 2pt minus 1pt}
%\setlength{\emergencystretch}{3em}  % prevent overfull lines 

\usepackage[T1]{fontenc}
\usepackage[utf8]{inputenc}
\linespread{1.05}

\usepackage{fancyhdr}
\pagestyle{fancy}
\usepackage{lastpage}
\renewcommand{\headrulewidth}{0.3pt}
\renewcommand{\footrulewidth}{0.0pt} 
\lhead{\footnotesize \textbf{Dr.~Furtado}}
\chead{}
\rhead{\footnotesize \emph{TSJ \textbar{} Homework: Descriptive Stats
01}}
%\lfoot{}
%\cfoot{\small \thepage/\pageref*{LastPage}}
%\rfoot{}

\fancypagestyle{firststyle}
{
\renewcommand{\headrulewidth}{0pt}%
   \fancyhf{}
   \fancyfoot[C]{\small \thepage/\pageref*{LastPage}}
}

%\def\labelitemi{--}
%\usepackage{enumitem}
%\setitemize[0]{leftmargin=25pt}
%\setenumerate[0]{leftmargin=25pt}


\usepackage{titlesec}

\titleformat*{\subsection}{\itshape}

\makeatletter
\@ifpackageloaded{hyperref}{}{%
\ifxetex
  \usepackage[setpagesize=false, % page size defined by xetex
              unicode=false, % unicode breaks when used with xetex
              xetex]{hyperref}
\else
  \usepackage[unicode=true]{hyperref}
\fi
}
\@ifpackageloaded{color}{
    \PassOptionsToPackage{usenames,dvipsnames}{color}
}{%
    \usepackage[usenames,dvipsnames]{color}
}
\makeatother
\hypersetup{breaklinks=true,
            bookmarks=true,
            pdfauthor={ ()},
             pdfkeywords = {},  
            pdftitle={TSJ \textbar{} Homework: Descriptive Stats 01},
            colorlinks=true,
            citecolor=blue,
            urlcolor=blue,
            linkcolor=magenta,
            pdfborder={0 0 0}}
\urlstyle{same}  % don't use monospace font for urls


\setcounter{secnumdepth}{0}





\usepackage{setspace}

\title{TSJ \textbar{} Homework: Descriptive Stats 01}
\author{Dr.~Furtado}
\date{}


\def\citeapos#1{\citeauthor{#1}'s (\citeyear{#1})}

% header includes!
\linespread{1.05}


\begin{document}  



\thispagestyle{plain} 

\begin{flushleft}\Large \bf TSJ \textbar{} Homework: Descriptive Stats
01  \end{flushleft}
	\vspace{1 mm}   
Dr.~Furtado \\
\emph{Department of Kinesiology, Cal State Northridge} \\
\texttt{\href{mailto:ovandef@csun.edu}{\nolinkurl{ovandef@csun.edu}}}   \\

% \blankline 
  

\hrule

\vspace{6 mm}
	


\hypertarget{learning-objectives}{%
\paragraph{Learning objectives}\label{learning-objectives}}

\begin{enumerate}
\def\labelenumi{\arabic{enumi}.}
\tightlist
\item
  create filters in JASP
\item
  calculate measures of central tendency and variability in JASP
\item
  differentiate between descriptive and inferential statistics
\item
  create and interpret histograms and boxplots
\end{enumerate}

\hypertarget{dataset-download-link-will-be-available-on-canvas}{%
\paragraph{Dataset: download link will be available on
Canvas}\label{dataset-download-link-will-be-available-on-canvas}}

We will use a modified version of the NFL Combine data set (ref needed).
The data set can be downloaded from this link:
\url{https://osf.io/tbgfh/}

Note that you will be required to apply/remove filters when completing
some of the required analysis for this assignment\footnote{Refer to this
  link for help: \url{https://bit.ly/37RPncf}}.

\begin{center}\rule{0.5\linewidth}{0.5pt}\end{center}

Open the data set in JASP and proceed to answer the following
questions\footnote{Questions are worth 10 points}:

\textbf{Question 1}

\textbf{Filter}: apply a filter to \texttt{Position} so that only
\texttt{Wide\ Receivers} (WR) and \texttt{Safety} (S) are included in
this analysis.

For the variables \texttt{Heightin} and \texttt{Weightlbs}, compute the
following: \texttt{mean}, \texttt{median}, \texttt{standard\ deviation},
\texttt{min}, \texttt{max}, and \texttt{sample\ size} (n).

Provide below the \textbf{Descriptive Statistics} Table:

\begin{table}[h]
    \centering
    \caption{Descriptive Statistics - DEMONSTRATION}
    \label{tab:descriptiveStatistics}
    {
        \begin{tabular}{lrrrr}
            \toprule
            \multicolumn{1}{c}{} & \multicolumn{2}{c}{Heightin} & \multicolumn{2}{c}{Weightlbs} \\
            \cline{2-3}\cline{4-5}
             & S & WR & S & WR  \\
            \cmidrule[0.4pt]{1-5}
            Valid & 25 & 43 & 25 & 43  \\
            Missing & 0 & 0 & 0 & 0  \\
            Mean & 71.920 & 72.674 & 206.320 & 203.651  \\
            Median & 72.000 & 73.000 & 207.000 & 202.000  \\
            Std. Deviation & 1.681 & 1.835 & 7.609 & 11.284  \\
            Minimum & 69.000 & 69.000 & 191.000 & 182.000  \\
            Maximum & 76.000 & 76.000 & 218.000 & 234.000  \\
            \bottomrule
        \end{tabular}
    }
\end{table}

\begin{center}\rule{0.5\linewidth}{0.5pt}\end{center}

\textbf{Question 2}

\textbf{Filter}: remove all filters before proceeding!

Considering the nature of both dependent variables used in Question 1
and assuming the distribution of scores for both variables are
\textbf{approximating normality}, which measure of central tendency
should be reported (mean, mode, or median)? Explain.

\begin{center}\rule{0.5\linewidth}{0.5pt}\end{center}

\textbf{Question 3}

In questions 1, you calculated the measures of central tendency and
variability, which fall under the category of descriptive statistics.
Discuss the difference between \textbf{descriptive statistics} and
\textbf{inferential statistics}. More complete answers will receive more
points.

\begin{center}\rule{0.5\linewidth}{0.5pt}\end{center}

\textbf{Question 4}

\textbf{Filter}: apply a filter to \texttt{Position} so that only
\texttt{Quarterbacks} (QB) are included in this analysis.

Using the variable \texttt{BroadJumpin}, calculate the \texttt{range},
\texttt{standard\ deviation} and the \texttt{IQR} for Quarterbacks (QB)
ONLY.

Provide below the \textbf{Descriptive Statistics} Table:

\begin{table}[h]
    \centering
    \caption{Descriptive Statistics - DEMONSTRATION}
    \label{tab:descriptiveStatistics}
    {
        \begin{tabular}{lrrrr}
            \toprule
            \multicolumn{1}{c}{} & \multicolumn{2}{c}{Heightin} & \multicolumn{2}{c}{Weightlbs} \\
            \cline{2-3}\cline{4-5}
             & S & WR & S & WR  \\
            \cmidrule[0.4pt]{1-5}
            Valid & 25 & 43 & 25 & 43  \\
            Missing & 0 & 0 & 0 & 0  \\
            Mean & 71.920 & 72.674 & 206.320 & 203.651  \\
            Median & 72.000 & 73.000 & 207.000 & 202.000  \\
            Std. Deviation & 1.681 & 1.835 & 7.609 & 11.284  \\
            Minimum & 69.000 & 69.000 & 191.000 & 182.000  \\
            Maximum & 76.000 & 76.000 & 218.000 & 234.000  \\
            \bottomrule
        \end{tabular}
    }
\end{table}

\begin{center}\rule{0.5\linewidth}{0.5pt}\end{center}

\textbf{Question 5}

\textbf{Filter}: remove all filters before proceeding!

In Question 1, you were asked to calculate the standard deviation of
height (inches) and weight (lbs). Suppose you want to compare the
\texttt{standard\ deviations} of \texttt{height} and \texttt{weight}.
How the two standard deviations compare?

\begin{center}\rule{0.5\linewidth}{0.5pt}\end{center}

\textbf{Question 6}

When calculating the sample variance and standard deviation, JASP uses
\texttt{N-1} in the denominator (see below). In your own words, explain
why \texttt{N-1} is used instead of \texttt{N}.

Standard Deviation equation for the sample:

\(\sigma = \sqrt{\frac{\sum (x - \mu)^2}{N-1}}\)

Variance equation for the sample:

\(s^2 = \frac{\sum (x - \bar{x})^2}{N - 1}\)

\begin{center}\rule{0.5\linewidth}{0.5pt}\end{center}

\textbf{Question 7}

The \texttt{variance} and the \texttt{standard\ deviation} are the two
most common measures of variability reported in research manuscripts.
Let's say the manuscript you submitted for publication was returned by
the editor. The editor-in-chief has asked you to report either the
\texttt{variance} OR the \texttt{standard\ deviation}. Which one would
you pick and why?

\begin{center}\rule{0.5\linewidth}{0.5pt}\end{center}

\textbf{Question 8}

\textbf{Filter}: apply a filter to \texttt{Status} so that \textbf{only}
\texttt{Year\ 2} is included in this analysis.

The variable ``Status'' refers to players who were either tested during
the first or second year. Run an analysis to calculate the
\texttt{mean}, and \texttt{standard\ deviation} for the variable
\texttt{Shuttle}.

Provide below the \textbf{Descriptive Statistics} Table:

\begin{table}[h]
    \centering
    \caption{Descriptive Statistics - DEMONSTRATION}
    \label{tab:descriptiveStatistics}
    {
        \begin{tabular}{lr}
            \toprule
             & Shuttle  \\
            \cmidrule[0.4pt]{1-2}
            Valid & 78  \\
            Missing & 42  \\
            Mean & 4.433  \\
            Std. Deviation & 0.266  \\
            \bottomrule
        \end{tabular}
    }
\end{table}

\begin{center}\rule{0.5\linewidth}{0.5pt}\end{center}

\textbf{Question 9}

Create a histogram for the variable \texttt{Shuttle} and add the density
line to it.

Provide the histogram create by JASP below and state whether the
distribution of scores for \texttt{Shuttle} appears to be deviating or
approximating normality. In this particular case, disregard other
sources of normality (skewness, kurtosis, QQ-plots, Shapiro-Wilk test,
etc.).

\begin{center}\rule{0.5\linewidth}{0.5pt}\end{center}

Provide below the \textbf{histogram} graph:

\textbf{Question 10}

Create a histogram for \texttt{Shuttle} using \texttt{ggplot2} as the
color palette. Did the creation of the boxplot reveal any outliers?
Explain.

Provide below the \textbf{boxblot} graph:

\begin{center}\rule{0.5\linewidth}{0.5pt}\end{center}

\textbf{Answer Sheet}: Students completing this assignment for grade may
copy the content below and paste it in a word processor (e.g., MS Word,
Google Docs, etc.)

Copy content from next line below:

\begin{center}\rule{0.5\linewidth}{0.5pt}\end{center}

Institution Name

Course Name

Instructor name

Student Name

Question 1

Answer:

Any relevant tables/graphs

\begin{center}\rule{0.5\linewidth}{0.5pt}\end{center}

Question 2

Answer:

Any relevant tables/graphs

\begin{center}\rule{0.5\linewidth}{0.5pt}\end{center}

Question 3

Answer:

Any relevant tables/graphs




\end{document}

\makeatletter
\def\@maketitle{%
  \newpage
%  \null
%  \vskip 2em%
%  \begin{center}%
  \let \footnote \thanks
    {\fontsize{18}{20}\selectfont\raggedright  \setlength{\parindent}{0pt} \@title \par}%
}
%\fi
\makeatother
