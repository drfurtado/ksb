\documentclass[11pt,]{article}
\usepackage[margin=1in]{geometry}
\newcommand*{\authorfont}{\fontfamily{phv}\selectfont}
\usepackage[]{mathpazo}
\usepackage{abstract}
\renewcommand{\abstractname}{}    % clear the title
\renewcommand{\absnamepos}{empty} % originally center
\newcommand{\blankline}{\quad\pagebreak[2]}

\providecommand{\tightlist}{%
  \setlength{\itemsep}{0pt}\setlength{\parskip}{0pt}} 
\usepackage{longtable,booktabs}

\usepackage{parskip}
\usepackage{titlesec}
\titlespacing\section{0pt}{12pt plus 4pt minus 2pt}{6pt plus 2pt minus 2pt}
\titlespacing\subsection{0pt}{12pt plus 4pt minus 2pt}{6pt plus 2pt minus 2pt}

\usepackage{titling}
\setlength{\droptitle}{-.25cm}

%\setlength{\parindent}{0pt}
%\setlength{\parskip}{6pt plus 2pt minus 1pt}
%\setlength{\emergencystretch}{3em}  % prevent overfull lines 

\usepackage[T1]{fontenc}
\usepackage[utf8]{inputenc}
\linespread{1.05}

\usepackage{fancyhdr}
\pagestyle{fancy}
\usepackage{lastpage}
\renewcommand{\headrulewidth}{0.3pt}
\renewcommand{\footrulewidth}{0.0pt} 
\lhead{\footnotesize \textbf{Dr.~Furtado}}
\chead{}
\rhead{\footnotesize \emph{KIN 610 \textbar{} Final Project -
Instructions}}
%\lfoot{}
%\cfoot{\small \thepage/\pageref*{LastPage}}
%\rfoot{}

\fancypagestyle{firststyle}
{
\renewcommand{\headrulewidth}{0pt}%
   \fancyhf{}
   \fancyfoot[C]{\small \thepage/\pageref*{LastPage}}
}

%\def\labelitemi{--}
%\usepackage{enumitem}
%\setitemize[0]{leftmargin=25pt}
%\setenumerate[0]{leftmargin=25pt}


\usepackage{titlesec}

\titleformat*{\subsection}{\itshape}

\makeatletter
\@ifpackageloaded{hyperref}{}{%
\ifxetex
  \usepackage[setpagesize=false, % page size defined by xetex
              unicode=false, % unicode breaks when used with xetex
              xetex]{hyperref}
\else
  \usepackage[unicode=true]{hyperref}
\fi
}
\@ifpackageloaded{color}{
    \PassOptionsToPackage{usenames,dvipsnames}{color}
}{%
    \usepackage[usenames,dvipsnames]{color}
}
\makeatother
\hypersetup{breaklinks=true,
            bookmarks=true,
            pdfauthor={ ()},
             pdfkeywords = {},  
            pdftitle={KIN 610 \textbar{} Final Project - Instructions},
            colorlinks=true,
            citecolor=blue,
            urlcolor=blue,
            linkcolor=magenta,
            pdfborder={0 0 0}}
\urlstyle{same}  % don't use monospace font for urls


\setcounter{secnumdepth}{0}

\usepackage{longtable}




\usepackage{setspace}

\title{KIN 610 \textbar{} Final Project - Instructions}
\author{Dr.~Furtado}
\date{}


\def\citeapos#1{\citeauthor{#1}'s (\citeyear{#1})}

% header includes!
\linespread{1.05}


\begin{document}  



\thispagestyle{plain} 

\begin{flushleft}\Large \bf KIN 610 \textbar{} Final Project -
Instructions  \end{flushleft}
	\vspace{1 mm}   
Dr.~Furtado \\
\emph{Department of Kinesiology, Cal State Northridge} \\
\texttt{\href{mailto:ovandef@csun.edu}{\nolinkurl{ovandef@csun.edu}}}   \\

% \blankline 
  

\hrule

\vspace{6 mm}
	


\hypertarget{learning-objectives}{%
\paragraph{Learning objectives}\label{learning-objectives}}

\begin{enumerate}
\def\labelenumi{\arabic{enumi}.}
\tightlist
\item
  Identify variables from a research question statement
\item
  Formulate hypotheses from a research question statement
\item
  Identify the appropriate statistical procedure to test the formulated
  hypothesis
\item
  Conduct statistical analysis in SPSS
\item
  Interpret the results of the data analysis
\item
  Create tables and figures to illustrate the findings
\item
  Prepare a research report
\end{enumerate}

\hypertarget{dataset}{%
\paragraph{Dataset}\label{dataset}}

\begin{enumerate}
\def\labelenumi{\arabic{enumi}.}
\tightlist
\item
  \href{https://osf.io/bf58v/}{Link} to download the data set
\item
  \href{https://osf.io/sg46v/}{Link} to download the data codebook
\end{enumerate}

The date for this project come from the
\href{https://en.wikipedia.org/wiki/NFL_Scouting_Combine}{NFL Scouting
Combine}. The NFL Combine is held prior to the draft every year, testing
players in the 40 yard dash, vertical jump, bench press, broad jump,
shuttle, and three cone drill.

The description of each drill can be
\href{https://nflcombineresults.com/nfl-combine-drills-101-what-each-drill-measures/}{found
here} and
\href{https://www.espn.com/nfl/draft2018/story/_/id/22587931/guide-nfl-draft-combine-drills-todd-mcshay-numbers-know-40-yard-dash-short-shuttle-bench-press}{here}.
Please,
\href{https://protips.dickssportinggoods.com/sports-and-activities/football/football-101-football-positions-and-their-roles}{click
here} to learn about football positions.

\begin{quote}
This data set CANNOT be considered a true random sample since not
everyone from the study population had the same chance to be included in
the sample.
\end{quote}

For the final project, I have modified the data set in the following
ways:

\begin{enumerate}
\def\labelenumi{\arabic{enumi}.}
\item
  Extra variables were created for each dependent variable and random
  numbers were generated for these variables. Thus, the data for the
  ``pos-test'' variables are fabricated.
\item
  Scores were also fabricated using the test's population mean and
  standard deviation for the Wonderlic\footnote{Wonderlic test -
    Wikipedia." \url{https://en.wikipedia.org/wiki/Wonderlic_test}.
    Accessed 21 Oct.~2020.} variable. Due to privacy, the data for this
  variable are not publicly available.
\end{enumerate}

\begin{center}\rule{0.5\linewidth}{0.5pt}\end{center}

\hypertarget{steps}{%
\paragraph{Steps}\label{steps}}

You will be given a research question and a data set. Then, after
running the appropriate statistical model, proceed to write the report.

You will be asked to:

\hypertarget{identify-the-dependent-and-independent-variables}{%
\subparagraph{1. Identify the dependent and independent
variable(s)}\label{identify-the-dependent-and-independent-variables}}

Once you are given access to the research question and the data set,
identify the Independent and Dependent variables that will be part of
your analysis:

\begin{itemize}
\item
  For the DV, identify whether it is continuous or discrete (nominal or
  ordinal).
\item
  For the IV, identify the levels associated with it (i.,e. sex - two
  levels; males and females)
\item
  Whenever applicable, run the normality plots with tests to verify
  whether the distribution of scores for the dependent variable is
  approximating or deviating from normality. Use this information to
  decide whether you will need to use a parametric or non-parametric
  test (i., e. independent-samples t-test vs Mann-Whitney test).
\end{itemize}

\begin{center}\rule{0.5\linewidth}{0.5pt}\end{center}

\hypertarget{formulate-the-hypothesisses-for-each-of-the-dependent-variables}{%
\subparagraph{2. Formulate the hypothesis(ses) for each of the dependent
variable(s)}\label{formulate-the-hypothesisses-for-each-of-the-dependent-variables}}

A hypothesis for each of the DVs must be formulated. First, state your
prediction; i.e.,
\texttt{10-year-old\ girls\ will\ overperform\ 10-year-old\ boys\ on\ the\ skill\ of\ skipping}.

Next, you must decide whether you will be testing your hypothesis using
the one-tailed or the two-tailed test. Once this is decided, you are
ready to state the Ho and the Ha:

\begin{itemize}
\tightlist
\item
  Ho: 10-year-old girls and boys perform similarly on the skill of
  skipping
\item
  Ha: 10-year-old girls and boys will perform differently on the skill
  of skipping
\end{itemize}

Notice above that Ha was stated as non-directional (two-tailed) because,
even though we predicted that girls will do better than boys, we are
uncertain of this prediction. Thus, the recommendation is to choose the
two-tailed test.

Recall that you should only use a directional hypothesis (one-tailed) if
you have strong evidence of the direction of the effect (refer to our
textbook about one- and two-tailed tests).

\begin{center}\rule{0.5\linewidth}{0.5pt}\end{center}

\hypertarget{choose-the-appropriate-statistical-model-to-test-the-hypothesisses}{%
\subparagraph{3. Choose the appropriate statistical model to test the
hypothesis(ses)}\label{choose-the-appropriate-statistical-model-to-test-the-hypothesisses}}

State the statistical procedure (i.e., ANOVA, Spearman rho) you selected
to test the hypothesis(ses) and explain the rationale for choosing the
procedure(s).

Recall that the selection of the procedure will depend on several
factors, including but not limited to, \texttt{1)} the nature of the
dependent variable (continuous, discrete), \texttt{2)} the level
(nominal, ordinal, scale), \texttt{3)} the number of DVs/IVs. In
addition, because the sample you are using is not a
\texttt{true\ random\ sample}, normality cannot be assumed by default.
You must run the normality plots with tests for each dependent variable
and report it appropriately.

Note that you must chose a statistical model for each research
hypothesis \texttt{(Ha)} formulated. For instance, if\ldots{}

\begin{center}\rule{0.5\linewidth}{0.5pt}\end{center}

\hypertarget{run-the-chosen-model-using-jasp}{%
\subparagraph{4. Run the chosen model using
JASP}\label{run-the-chosen-model-using-jasp}}

Use JASP to open the data set and run the statistical model selected in
\texttt{\#3} above.

Note that the relevant graphs/tables generated by JASP should be used to
support your \texttt{write-up}.

\begin{center}\rule{0.5\linewidth}{0.5pt}\end{center}

\hypertarget{data-analysis-and-interpretation}{%
\subparagraph{5. Data Analysis and
Interpretation}\label{data-analysis-and-interpretation}}

Under this section, you will be asked to interpret the results based on
the output generated by JASP.

\textbf{This is the most important section of your report. Ensure to be
throughout when writing this section.}

\begin{quote}
Hint: Please download and read
\href{https://www.ncbi.nlm.nih.gov/pmc/articles/PMC4062332/pdf/biochem-med-22-1-15-3.pdf}{this
article} that covers the best practices of Methods/Results/Conclusion
write-ups.
\end{quote}

Include the following and in the sequence presented below:

\begin{enumerate}
\def\labelenumi{\arabic{enumi}.}
\item
  Statistical Analysis

  \begin{itemize}
  \tightlist
  \item
    Refer to the NCBI article (link above) to learn what to include
    here.
  \end{itemize}
\item
  Results\footnote{Ensure to phrase your results following the APA Style
    → \url{https://bit.ly/2HirVLv}}
\end{enumerate}

\begin{enumerate}
\def\labelenumi{\alph{enumi}.}
\item
  You will be required to add \texttt{at\ least\ two} tables and
  \texttt{one\ graph} to enhance the \texttt{Results} section. Tables
  and graphs must come from JASP and be related to the statistical model
  used to test your hypothesis(es).

  \begin{itemize}
  \item
    Descriptive Statistics (AKA demographics)
  \item
    Results table related to analysis you used
  \end{itemize}
\item
  Conclusions

  \begin{itemize}
  \tightlist
  \item
    Refer to the NCBI article (link above) to learn what to include
    here.
  \end{itemize}
\item
  Limitations

  \begin{itemize}
  \tightlist
  \item
    Refer to the NCBI article (link above) to learn what to include
    here.
  \end{itemize}
\end{enumerate}

\begin{center}\rule{0.5\linewidth}{0.5pt}\end{center}

\hypertarget{references-following-the-6th-apa-style}{%
\subparagraph{6. References (following the 6th APA
Style)}\label{references-following-the-6th-apa-style}}

All sources consulted must be properly cite (in-text citation following
APA Style) and the sources must be listed under \texttt{References}.

\begin{center}\rule{0.5\linewidth}{0.5pt}\end{center}

\hypertarget{appendices}{%
\paragraph{Appendices}\label{appendices}}

\textbf{A. Correlation reference criteria}

When evaluating the size of a bivariate correlation, please use Cohen
(1988)

\begin{longtable}[]{@{}ll@{}}
\toprule
Coeffficient Value & Strength of Association \\ \addlinespace
\midrule
\endhead
0.1 \textless{} \emph{r} \textless{} .3 & small
correlation \\ \addlinespace
0.3 \textless{} \emph{r} \textless{} .5 & medium/moderate
correlation \\ \addlinespace
\emph{r} \textgreater{} .5 & Large/strong correlation \\ \addlinespace
\bottomrule
\end{longtable}

\begin{center}\rule{0.5\linewidth}{0.5pt}\end{center}

\textbf{B. Research Questions}\footnote{If a \texttt{parametric} test is
  selected, do not forget to run the normality test.}

\textbf{\emph{Research Question \#1}}

It has been published in the press that players from
\texttt{Iowa\ State} work harder during practice compared to players
from any other college teams in the country.

Since you had access to players from another university in the State of
Iowa \texttt{(University\ of\ Iowa)}, you decided to test this
hypothesis.

As a group, do players from \texttt{Iowa\ State\ (70)} perform better
than players from \texttt{Iowa\ University\ (69)} on the
\texttt{Bench\ Press\ (post)} test? How about for
\texttt{vertical\ leap\ (post)}?

\begin{quote}
You will need to Filter cases to perform this analysis so that only
values \texttt{69} and \texttt{70} is selected for \texttt{college}. If
a parametric test is selected/checked.
\end{quote}

\textbf{\emph{Research Question \#2}}

Research has shown that in the general population there is a
\texttt{negative} and \texttt{moderate\ to\ high} correlation between
\texttt{weight} and \texttt{performance\ on\ the\ test\ of}Broad Jump`.
In other words, the heavier the person, the poorer the performance on
the test, and vice-versa.

Assume that you are specially interested in football players that play
as \texttt{Defensive\ Ends\ (3)}. Is there a relationship between
\texttt{Weight} and \texttt{Broad\ Jump\ (pre)} scores among
\texttt{Defensive\ Ends\ (3)}?

\begin{quote}
You will need to Filter cases to perform this analysis so that only the
value \texttt{3} is selected/checked for \texttt{position}.
\end{quote}

\textbf{\emph{Research Question \#3}}

Defensive players are known to be stronger than those playing on other
positions. But how about if we compare defense players among themselves
from different positions?

Are Defensive \texttt{Tackles\ (4)}, \texttt{Linebackers\ (10)}, and
\texttt{Cornerbacks\ (2)} different when it comes to
\texttt{bench\ press\ (pre)} scores?

\begin{quote}
You will need to Filter cases to perform this analysis so that only the
valuea \texttt{4}, \texttt{10} and \texttt{2} are selected/checked for
\texttt{position}.
\end{quote}

\textbf{\emph{Research Question \#4}}

As an athletic trainer working for the NFL Scouting Combine, you decided
to test the effectiveness of a program you developed to improve players'
agility. If deemed effective, the program could eventually be sold to
NFL professional teams.

To test the effectiveness of the program, you invited players from
\texttt{Ohio\ State\ University} to participate in this 2-week program
(3 hours everyday). At the end of the 2-week period, players were
re-tested on the \texttt{20-yard\ Shuttle} and the
\texttt{3-cone\ Drill}.

Do players from \texttt{Ohio\ State\ University\ (127)} improve their
scores on the \texttt{20-yard\ Shuttle} from pre to post-test? How about
for the \texttt{3-cone\ Drill}?

\begin{quote}
You will need to Filter cases to perform this analysis so that only the
value \texttt{127} is selected/checked for \texttt{college}.
\end{quote}

\textbf{\emph{Research Question \#5}}

There is evidence that \texttt{Wide\ Receivers} and
\texttt{Running\ Backs} have a higher incidence of concussion compared
to players from other football positions. Over the years, this may
negatively affect the players' cognitive ability. For instance, how
would they compare to Quarterbacks, who arguably are less prone to
suffer concussions during a football match.

How do players playing as \texttt{Wide\ Receivers\ (22)},
\texttt{Running\ Backs\ (18)} and \texttt{Quarterbacks\ (17)} compare on
the Wonderlic scores? For this analysis, please, ONLY use the data for
2020 data?

\begin{quote}
You will need to Filter cases to perform this analysis so that only the
values \texttt{17}, \texttt{18} and \texttt{22} are selected/checked for
\texttt{position}.
\end{quote}

\textbf{\emph{Research Question \#6}}

Quarterbacks must have excellent decision-making skills and act quickly
under pressure during game plays.

Since the Wonderlic\footnote{\url{https://en.wikipedia.org/wiki/Wonderlic_test}}
test assesses cognitive ability under pressure, an interesting question
is whether QBs are above average when it comes to cognitive ability.

According to the test developers, the average (mean) score on the
Wonderlic test is 20 and the median score is (19).

Do QBs tested in 2018, 2019, and 2019 perform better than the general
population on the Wonderlic Cognitive ability test?

\begin{quote}
You will need to Filter cases to perform this analysis so that only the
value \texttt{17}, is selected/checked for \texttt{position}.
\end{quote}

\textbf{\emph{Research Question \#7}}

It has been
\href{https://www.cbssports.com/nfl/news/nfl-draft-combine-the-highest-and-lowest-wonderlic-test-scores-ever-recorded/}{reported}
that as a group, \texttt{Offensive\ Tackles} perform better than
\texttt{Full\ Backs} on the \texttt{Wonderlic\ test}.

Test the hypothesis that OT (15) players tested in 2018, 2019, and 2020
perform differently than FB (6) players on the Wonderlic Cognitive
Ability test.

\begin{quote}
You will need to Filter cases to perform this analysis so that only the
values \texttt{5} and \texttt{15} are selected/checked for
\texttt{position}.
\end{quote}

\textbf{\emph{Research Question \#8}}

n the general population, there is a strong positive correlation between
\texttt{weight} and \texttt{speed}. In other words, the heavier the
player the slower the individual is, and vice-versa.

Considering that defensive players do train speed during football
practices, it would be interesting to verify whether college players
have a similar pattern compared to the general population.

Is there a correlation between \texttt{weight} and \texttt{speed} among
football \texttt{Defensive\ Tackles\ (4)}? How about
\texttt{Defensive\ Ends\ (3)}?

\begin{quote}
For this analysis, use the variables \texttt{Weight} and
\texttt{40-yard\ Dash\ Pretest}, and filter cases so that only positions
\texttt{3} and \texttt{4} are selected/checked.
\end{quote}

\textbf{\emph{Research Question \#9}}

As a Motor Behaviorist, you work for the NFL Scouting Combine and see
the opportunity to collect data and test some of the research questions
you have in mind.

In 2020, you designed a program to help players improve their
\texttt{speed}.

\texttt{Quarterbacks\ (17)} from all attending colleges were selected to
be part of the intervention. Test the hypothesis that the players would
improve from pre to post-test on the \texttt{40-yard\ Shuttle}.

\begin{quote}
For this analysis, use the variables \texttt{40-yard\ Shuttle}, and
filter cases so that only position \texttt{17} is selected/checked.
Also, note that this study involves only players who were tested in
\texttt{2020}.
\end{quote}

\textbf{\emph{Research Question \#10}}

As a Motor Behaviorist, you work for the NFL Scouting Combine and see
the opportunity to collect data and test some of the research questions
you have in mind.

In 2020, you designed a program to help players improve their
\texttt{speed}.

\texttt{Running\ Backs\ (18)} from all attending colleges were selected
to be part of the intervention. Test the hypothesis that the players
would improve from pre to post-test on the \texttt{40-yard\ Shuttle}.

\begin{quote}
For this analysis, use the variables \texttt{40-yard\ Shuttle}, and
filter cases so that only position \texttt{18} is selected/checked.
Also, note that this study involves only players who were tested in
\texttt{2020}.
\end{quote}

\textbf{\emph{Research Question \#11}}

As a Motor Behaviorist, you work for the NFL Scouting Combine and see
the opportunity to collect data and test some of the research questions
you have in mind.

In 2020, you designed a program to help players improve their
\texttt{agility}.

\texttt{Quarterbacks\ (17)} from all attending colleges were selected to
be part of the intervention. Test the hypothesis that the players would
improve from pre to post-test on the \texttt{3-cone\ Drill}.

\begin{quote}
For this analysis, use the variables \texttt{3-cone\ Drill}, and filter
cases so that only position \texttt{17} is selected/checked. Also, note
that this study involves only players who were tested in \texttt{2020}.
\end{quote}

\textbf{\emph{Research Question \#12}}

As a Motor Behaviorist, you work for the NFL Scouting Combine and see
the opportunity to collect data and test some of the research questions
you have in mind.

In 2020, you designed a program to help players improve their
\texttt{agility}.

\texttt{Running\ Backs\ (18)} from all attending colleges were selected
to be part of the intervention. Test the hypothesis that the players
would improve from pre to post-test on the \texttt{3-cone\ Drill}.

\begin{quote}
For this analysis, use the variables \texttt{3-cone\ Drill}, and filter
cases so that only position \texttt{18} is selected/checked. Also, note
that this study involves only players who were tested in \texttt{2020}.
\end{quote}




\end{document}

\makeatletter
\def\@maketitle{%
  \newpage
%  \null
%  \vskip 2em%
%  \begin{center}%
  \let \footnote \thanks
    {\fontsize{18}{20}\selectfont\raggedright  \setlength{\parindent}{0pt} \@title \par}%
}
%\fi
\makeatother
